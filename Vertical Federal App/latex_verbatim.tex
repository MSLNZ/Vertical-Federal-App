%%=======================================================================
\documentclass[IANZ]{MSLCalCert}
\usepackage{parskip}
\usepackage{upgreek}
\usepackage{longtable,dcolumn}
\usepackage{ragged2e}
\usepackage{graphics}
\graphicspath{ {C:/Users/c.young/Pictures}}


\begin{document}
\date{27 July 2022}
\reportnumber{Length/2022/1226}
\serial{Set Serial Number:302416 (S055)}
\fileref{L0179/J00711}
\title{Report on the Calibration of a Set of Gauge Blocks}
\maketitlepage
\justifying
\section{Description}
The gauge block set is manufactured by Mitututo and contains eight metric tungsten carbide gauge blocks ranging in nominal size from 3.1 mm to 3.1 mm.
\section{Identification}
Figure 1 shows the exterior of the box housing the gauge block set.\par
\includegraphics[width=16cm]{C:/Users/c.young/Pictures/20210228_165132.jpg}
Figure 1: Box Exterior\pagebreak
\section{Client}
Metrology Calibration Services Limited, 130C Cambridge Road, Te Awamutu, New Zealand
\section{Date of Calibration}
13 May 22 to 16 May 22
\section{Method}
The deviation and variation in length of each gauge block was measured according to technical procedure MSLT.L.003.007 \emph{Gauge Blocks(<= 101.4 mm), Calibration by comparison}.  Measurements were taken in five positions on each gauge(at the centre and toward each corner).\par \emph{Deviation} is defined as the measured length minus the nominal length. \emph{Extreme deviation} is defined as either the maximum or minimum measured deviation, depending on which has the larger magnitude.  \emph{Variation} in length is defined as the difference between the maximum measured length and the minimum measured length. \emph{Limit Deviation} is defined as the permissible deviation at any point on the measuring face.
\section{Objectives}
To measure the centre deviation, extreme deviation and variation in length of each gauge block for compliance with:\par 
\begin{itemize}
\item BS EN ISO 3650:1999 Geometrical product specifications (GPS) - Length standards - Gauge blocks, Grade 1 classification.\par
\end{itemize}
The standard(s) listed here are referred to as "documentary standards" in this report.
\section{Conditions}
Gauge block measurements were made with the comparator platen in the temperature range $\SI{19.8}{\celsius}$ to $\SI{20.2}{\celsius}$.\pagebreak
\section{Results}
The centre deviation of the gauge blocks is given in Table 1. The extreme deviation in length is also shown for determining compliance with the documentary standard(s) for limit deviation. The variation in length of the gauge blocks is given in Table 2. The tolerance for the variation in length is also shown for determining compliance with the documentary standard(s) for variation in length.\par The tables also include results for compliance with the requirements of documentary standard(s) for each gauge block for the limit deviation and variation in length.A ‘P’ in the compliance columns states that the gauge block meets the requirements of documentary standard for the tested condition.Similarly, an ‘F’ in the compliance columns states that the gauge block does not meet the requirements of documentary standard for the tested condition. A ‘U’ in the compliance columns states that compliance to the documentary standard for the tested condition cannot be confirmed or refuted. The expanded measurement uncertainty is considered for all compliance outcomes.\par The results are rounded to the nearest 0.001 $\mu$m and are valid at a reference temperature of $\SI{20}{\celsius}$.\par

% Although not used here, it is also possible to have a \subsubsection{}
%\subsection{Open (male), SN 54673}

 
\newcommand\hd[1]{\multicolumn{1}{c}{#1}}
\newcommand\zz[1]{^{#1}\!\!}
\setlength\extrarowheight{1pt}
\setlength\tabcolsep{10pt}
\begin{longtable}{D{.}{.}{4}cD{.}{.}{4}D{.}{.}{4}D{.}{.}{4}c}

%Line 1
\multicolumn{6}{l}{ \text{Table1: Deviations for gauge block set xxxx.} }\\
%Line 2 blank
\multicolumn{6}{l}{ \text{} }\\
%Line 3 
\multicolumn{1}{c}{ \text{Nominal} } & \multicolumn{1}{c}{ \text{Serial} } &\multicolumn{1}{c}{ \text{Deviation} } &\multicolumn{1}{c}{ \text{Extreme} } &\multicolumn{1}{c}{ \text{Limit} } &\multicolumn{1}{c}{ \text{Compliance} } \\
%Line 4
\multicolumn{1}{c}{ \text{Length} } & \multicolumn{1}{c}{ \text{Number} } &\multicolumn{1}{c}{ \text{} } & \multicolumn{1}{c}{ \text{Deviation} } &\multicolumn{1}{c}{ \text{Deviation} } & \multicolumn{1}{c}{ \text{Outcome}}\\
%line 5
\multicolumn{1}{c}{ \text{(mm)}} & &\multicolumn{1}{c}{ \text{($\mu$m)}} &\multicolumn{1}{c}{ \text{($\mu$m)}} &\multicolumn{1}{c}{ \text{($\mu$m)}} &\\ \hline
 \endfirsthead
%Line 1
\multicolumn{6}{l}{ \text{Table 1(Continued): Deviations for gauge block set xxxx.} }\\
%Line 2 blank
\multicolumn{6}{l}{ \text{} }\\
%Line 3
\multicolumn{1}{c}{ \text{Nominal} } & \multicolumn{1}{c}{ \text{Serial} } &\multicolumn{1}{c}{ \text{Deviation} } &\multicolumn{1}{c}{ \text{Extreme} } &\multicolumn{1}{c}{ \text{Limit} } &\multicolumn{1}{c}{ \text{Compliance} } \\
%Line 4
\multicolumn{1}{c}{ \text{Length} } & \multicolumn{1}{c}{ \text{Number} } &\multicolumn{1}{c}{ \text{} } & \multicolumn{1}{c}{ \text{Deviation} } &\multicolumn{1}{c}{ \text{Deviation} } & \multicolumn{1}{c}{ \text{}}\\

\multicolumn{1}{c}{ \text{(mm)}} & &\multicolumn{1}{c}{ \text{($\mu$m)}} &\multicolumn{1}{c}{ \text{($\mu$m)}} &\multicolumn{1}{c}{ \text{($\mu$m)}} &\\ \hline
\endhead
\endfoot
\endlastfoot


1 & AAA & 0.0023 & -0.01 & -3.27 & P\\
1.1 & 05678 & 0.0025 & -9.80 & 0.14 & P\\
2.4 & ffffff & 0.0026 & -16.34 & 0.15 & P\\
13.5 & dddd & 0.0032 & -32.72 & 0.18 & P\\
21.1 & ddddd & 0.0054 & -65.67 & 0.31 & P\\
18 & fdfds & 0.011 & -98.66 & 0.62 & P\\
100 & fsds & 0.013 & -131.74 & 0.78 & P\\
100 & fdsd & 0.016 & -164.77 & 0.90 & P\\
7.8 & dsd & 0.017 & +162.15 & 0.99 & P\\
55 & fddf & 0.018 & +129.0 & 1.1 & P\\
66 & ddd & 0.018 & +95.9 & 1.1 & P\\
777 & dfd & 0.018 & +62.7 & 1.1 & P\\
80 & fd & 0.0032 & -32.72 & 0.18 & P\\
1 & dfdsf & 0.0054 & -65.67 & 0.31 & P\\
40 & fdssd & 0.011 & -98.66 & 0.62 & P\\
50 & fdsf & 0.013 & -131.74 & 0.78 & P\\
25.5 & fdfs & 0.016 & -164.77 & 0.90 & P\\
30 & fdssd & 0.017 & +162.15 & 0.99 & P\\
30.5 & fds & 0.018 & +129.0 & 1.1 & P\\
40.5 & fsd & 0.018 & +95.9 & 1.1 & P\\
20 & gfdfg & 0.018 & +62.7 & 1.1 & P\\

\end{longtable}
 
\begin{longtable}{D{.}{.}{4}cD{.}{.}{4}D{.}{.}{4}c}


%Line 1
\multicolumn{5}{l}{ \text{Table 2: Variation in length for gauge block set xxxx.} }\\
%Line 2 blank
\multicolumn{5}{l}{ \text{} }\\
%Line 3
\multicolumn{1}{c}{ \text{Nominal} } & \multicolumn{1}{c}{ \text{Serial} } &\multicolumn{1}{c}{ \text{Variation} } &\multicolumn{1}{c}{ \text{Tolerance on} } &\multicolumn{1}{c}{ \text{Compliance} } \\
%Line 4
\multicolumn{1}{c}{ \text{Length} } & \multicolumn{1}{c}{ \text{Number} } &\multicolumn{1}{c}{ \text{in Length} } &\multicolumn{1}{c}{ \text{Variation in Length} } & \multicolumn{1}{c}{ \text{Outcome}}
\\
\multicolumn{1}{c}{ \text{(mm)}} & \multicolumn{1}{c}{ \text{($\mu$m)}} &\multicolumn{1}{c}{ \text{($\mu$m)}} &\multicolumn{1}{c}{ \text{($\mu$m)}} &\\ \hline
 \endfirsthead
%Line 1
\multicolumn{5}{l}{ \text{Table 2(Continued): Variation in length for gauge block set xxxx.} }\\
%Line 2 blank
\multicolumn{5}{l}{ \text{} }\\
%Line 3
\multicolumn{1}{c}{ \text{Nominal} } & \multicolumn{1}{c}{ \text{Serial} } &\multicolumn{1}{c}{ \text{Variation} } &\multicolumn{1}{c}{ \text{Tolerance on} } &\multicolumn{1}{c}{ \text{Compliance} } \\
%Line 4
\multicolumn{1}{c}{ \text{Length} } & \multicolumn{1}{c}{ \text{Number} } &\multicolumn{1}{c}{ \text{in Length} } &\multicolumn{1}{c}{ \text{Variation in Length} } & \multicolumn{1}{c}{ \text{Outcome}}\\

\multicolumn{1}{c}{ \text{(mm)}} & \multicolumn{1}{c}{ \text{($\mu$m)}} &\multicolumn{1}{c}{ \text{($\mu$m)}} &\multicolumn{1}{c}{ \text{($\mu$m)}} &\\ \hline
\endhead
\endfoot
\endlastfoot


1 & AAA & 0.0023 & -0.01 & P\\
1.1 & 05678 & 0.0025 & -9.800 & P\\
2.4 & ffffff & 0.0026 & -16.34 & P\\
13.5 & dddd & 0.0032 & -32.72 & P\\
21.1 & ddddd & 0.0054 & -65.67 & P\\
18 & fdfds & 0.011 & -98.66 & P\\
100 & fsds & 0.013 & -131.74 & P\\
100 & fdsd & 0.016 & -164.77 & P\\
7.8 & dsd & 0.017 & +162.15 & P\\
55 & fddf & 0.018 & +129.0 & P\\
66 & ddd & 0.018 & +95.9 & P\\
777 & dfd & 0.018 & +62.7 & P\\
80 & fd & 0.0032 & -32.72 &  P\\
1 & dfdsf & 0.0054 & -65.67 & P\\
40 & fdssd & 0.011 & -98.66 & P\\
50 & fdsf & 0.013 & -131.74 & P\\
25.5 & fdfs & 0.016 & -164.77  & P\\
30 & fdssd & 0.017 & +162.15 & P\\
30.5 & fds & 0.018 & +129.0 & P\\
40.5 & fsd & 0.018 & +95.9 & P\\
20 & gfdfg & 0.018 & +62.7 & P\\

\end{longtable}

\section{Uncertainty}
The expanded measurement uncertainties for the deviation and variation in length measurements are given in the Table 3.
\setlength\tabcolsep{30pt}
\begin{longtable}{D{.}{.}{4}cD{.}{.}{4}D{.}{.}{4}}


%Line 1
\multicolumn{4}{l}{ \text{Table 3: Expanded Uncertainties for gauge block set xxxx.} }\\
%Line 2 blank
\multicolumn{4}{l}{ \text{} }\\
%Line 3
\multicolumn{1}{c}{ \text{Nominal} } & \multicolumn{1}{c}{ \text{Centre} } &\multicolumn{1}{c}{ \text{Extreme} } &\multicolumn{1}{c}{ \text{Variation} } 
\\
%Line 4
\multicolumn{1}{c}{ \text{Length} } & \multicolumn{1}{c}{ \text{Deviation} } &\multicolumn{1}{c}{ \text{Deviation} } &\multicolumn{1}{c}{ \text{in Length} }  \\

\multicolumn{1}{c}{ \text{(mm)}} & \multicolumn{1}{c}{ \text{($\mu$m)}} &\multicolumn{1}{c}{ \text{($\mu$m)}} &\multicolumn{1}{c}{ \text{($\mu$m)}} \\ \hline
 \endfirsthead
%Line 1
\multicolumn{4}{l}{ \text{Table 3 (Continued): Expanded Uncertainties for gauge block set xxxx.} }\\
%Line 2 blank
\multicolumn{4}{l}{ \text{} }\\
%Line 3
\multicolumn{1}{c}{ \text{Nominal} } & \multicolumn{1}{c}{ \text{Centre} } &\multicolumn{1}{c}{ \text{Extreme} } &\multicolumn{1}{c}{ \text{Variation} } \\
%Line 4
\multicolumn{1}{c}{ \text{Length} } & \multicolumn{1}{c}{ \text{Deviation} } &\multicolumn{1}{c}{ \text{Deviation} } &\multicolumn{1}{c}{ \text{in Length} }  \\

\multicolumn{1}{c}{ \text{(mm)}} & \multicolumn{1}{c}{ \text{($\mu$m)}} &\multicolumn{1}{c}{ \text{($\mu$m)}} &\multicolumn{1}{c}{ \text{($\mu$m)}} \\ \hline
\endhead
\endfoot
\endlastfoot

1 & AAA & 0.0023 & -0.01\\
1.1 & 05678 & 0.0025 & -9.800\\
2.4 & ffffff & 0.0026 & -16.34\\
13.5 & dddd & 0.0032 & -32.72\\
21.1 & ddddd & 0.0054 & -65.67\\
18 & fdfds & 0.011 & -98.66\\
100 & fsds & 0.013 & -131.74\\
100 & fdsd & 0.016 & -164.77\\
7.8 & dsd & 0.017 & +162.15\\
55 & fddf & 0.018 & +129.0\\
66 & ddd & 0.018 & +95.9\\
777 & dfd & 0.018 & +62.7\\
80 & fd & 0.0032 & -32.72\\
1 & dfdsf & 0.0054 & -65.67\\
40 & fdssd & 0.011 & -98.66\\
50 & fdsf & 0.013 & -131.74\\
25.5 & fdfs & 0.016 & -164.77\\
30 & fdssd & 0.017 & +162.15\\
30.5 & fds & 0.018 & +129.0\\
40.5 & fsd & 0.018 & +95.0\\
20 & gfdfg & 0.018 & +62.7\\

\end{longtable}

A coverage factor of 2.0 was used to calculate the expanded uncertainties at a level of confidence of approximately 95\%. The number of degrees of freedom associated with each measurement result was large enough to justify this coverage factor.

\paragraph{Note:} \referenceGUM
\sigA{L A Evergreen}{Length Standards}{Research Engineer}
\sigB{C M Young}{Length Standards}{Senior Research Scientist}
\chiefMetrologistDelegate{T J Stewart}
\signaturesAB
\end{document}
